\section{文献综述}

\subsection{国内外研究现状}
近年来,随着计算机视觉和移动机器人技术的快速发展,基于视觉的同步定位与建图(SLAM)算法成为了研究热点。

在国外,Smith等人\cite{1}提出了基于特征点的视觉里程计方法,显著提高了机器人在复杂环境下的定位精度。随后,Johnson等人提出了一种基于直接法的视觉SLAM框架,有效克服了特征提取困难的问题。

在国内,李明等人针对动态环境下的视觉SLAM问题进行了深入研究,提出了一种基于语义信息的动态物体剔除算法。王强等人则关注由于光照变化引起的定位漂移问题,提出了一种光照鲁棒的视觉特征描述子。

\subsection{现有方法的不足}
尽管现有的视觉SLAM算法在许多场景下取得了成功,但仍存在以下不足:
\begin{itemize}
    \item 在纹理能够缺失的场景下,基于特征点的方法容易失效。
    \item 大多数算法假设环境是静态的,无法有效应对动态场景。
    \item 实时性与建图精度之间的平衡仍需进一步优化。
\end{itemize}

\subsection{本文的研究内容}
针对上述问题,本文提出了一种改进的视觉SLAM算法。主要研究内容包括:
\begin{enumerate}
    \item 提出一种新的特征提取策略,增强算法在弱纹理环境下的鲁棒性。
    \item 引入语义信息辅助定位,提高算法在动态场景下的稳定性。
    \item 设计高效的后端优化算法,在保证精度的同时提升系统的实时性。
\end{enumerate}
