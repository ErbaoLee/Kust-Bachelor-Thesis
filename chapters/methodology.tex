\section{研究方法}

\subsection{系统框架}
本文提出的视觉SLAM系统框架如图 \ref{fig:system_framework} 所示。系统主要由前端视觉里程计、后端优化、回环检测和建图四个模块组成。

\begin{figure}[htbp]
    \centering
    % 使用简单的文本框模拟系统框架图,实际使用时请替换为真实图片
    \framebox[0.8\textwidth][c]{\rule{0pt}{3cm}在此处插入系统框架图}
    \caption{系统框架示意图}
    \label{fig:system_framework}
\end{figure}

\subsection{前端视觉里程计}
前端视觉里程计负责估计相邻图像帧之间的相机运动。本文采用光流法进行特征跟踪,以提高系统的实时性。

假设 $t$ 时刻的图像为 $I_t$,特征点位置为 $(u, v)$,则光流方程可以表示为:
\begin{equation}
    I_x u + I_y v + I_t = 0
\end{equation}
其中,$I_x, I_y, I_t$ 分别为图像在 $x, y, t$ 方向的偏导数。

\subsection{后端优化}
后端优化模块用于消除前端累积的漂移误差。本文采用基于图优化的方法,构建位姿图(Pose Graph)进行全局优化。

目标函数定义为:
\begin{equation}
    E(\mathbf{x}) = \sum_{ij} \mathbf{e}_{ij}^T \Omega_{ij} \mathbf{e}_{ij}
\end{equation}
其中,$\mathbf{x}$ 为状态变量,$\mathbf{e}_{ij}$ 为误差项,$\Omega_{ij}$ 为信息矩阵。

\subsection{回环检测}
回环检测模块通过识别机器人是否回到曾经访问过的位置,来消除累积误差。本文采用基于词袋模型(Bag of Words)的方法进行回环检测。
